\documentclass[10pt,a4paper,onecolumn]{article}
\usepackage{marginnote}
\usepackage{graphicx}
\usepackage{xcolor}
\usepackage{authblk,etoolbox}
\usepackage{titlesec}
\usepackage{calc}
\usepackage{tikz}
\usepackage{hyperref}
\hypersetup{colorlinks,breaklinks=true,
            urlcolor=[rgb]{0.0, 0.5, 1.0},
            linkcolor=[rgb]{0.0, 0.5, 1.0}}
\usepackage{caption}
\usepackage{tcolorbox}
\usepackage{amssymb,amsmath}
\usepackage{ifxetex,ifluatex}
\usepackage{seqsplit}
\usepackage{xstring}

\usepackage{float}
\let\origfigure\figure
\let\endorigfigure\endfigure
\renewenvironment{figure}[1][2] {
    \expandafter\origfigure\expandafter[H]
} {
    \endorigfigure
}


\usepackage{fixltx2e} % provides \textsubscript
\usepackage[
  backend=biber,
%  style=alphabetic,
%  citestyle=numeric
]{biblatex}
\bibliography{paper.bib}

% --- Splitting \texttt --------------------------------------------------

\let\textttOrig=\texttt
\def\texttt#1{\expandafter\textttOrig{\seqsplit{#1}}}
\renewcommand{\seqinsert}{\ifmmode
  \allowbreak
  \else\penalty6000\hspace{0pt plus 0.02em}\fi}


% --- Pandoc does not distinguish between links like [foo](bar) and
% --- [foo](foo) -- a simplistic Markdown model.  However, this is
% --- wrong:  in links like [foo](foo) the text is the url, and must
% --- be split correspondingly.
% --- Here we detect links \href{foo}{foo}, and also links starting
% --- with https://doi.org, and use path-like splitting (but not
% --- escaping!) with these links.
% --- Another vile thing pandoc does is the different escaping of
% --- foo and bar.  This may confound our detection.
% --- This problem we do not try to solve at present, with the exception
% --- of doi-like urls, which we detect correctly.


\makeatletter
\let\href@Orig=\href
\def\href@Urllike#1#2{\href@Orig{#1}{\begingroup
    \def\Url@String{#2}\Url@FormatString
    \endgroup}}
\def\href@Notdoi#1#2{\def\tempa{#1}\def\tempb{#2}%
  \ifx\tempa\tempb\relax\href@Urllike{#1}{#2}\else
  \href@Orig{#1}{#2}\fi}
\def\href#1#2{%
  \IfBeginWith{#1}{https://doi.org}%
  {\href@Urllike{#1}{#2}}{\href@Notdoi{#1}{#2}}}
\makeatother

\newlength{\cslhangindent}
\setlength{\cslhangindent}{1.5em}
\newlength{\csllabelwidth}
\setlength{\csllabelwidth}{3em}
\newenvironment{CSLReferences}[3] % #1 hanging-ident, #2 entry spacing
 {% don't indent paragraphs
  \setlength{\parindent}{0pt}
  % turn on hanging indent if param 1 is 1
  \ifodd #1 \everypar{\setlength{\hangindent}{\cslhangindent}}\ignorespaces\fi
  % set entry spacing
  \ifnum #2 > 0
  \setlength{\parskip}{#2\baselineskip}
  \fi
 }%
 {}
\usepackage{calc}
\newcommand{\CSLBlock}[1]{#1\hfill\break}
\newcommand{\CSLLeftMargin}[1]{\parbox[t]{\csllabelwidth}{#1}}
\newcommand{\CSLRightInline}[1]{\parbox[t]{\linewidth - \csllabelwidth}{#1}}
\newcommand{\CSLIndent}[1]{\hspace{\cslhangindent}#1}

% --- Page layout -------------------------------------------------------------
\usepackage[top=3.5cm, bottom=3cm, right=1.5cm, left=1.0cm,
            headheight=2.2cm, reversemp, includemp, marginparwidth=4.5cm]{geometry}

% --- Default font ------------------------------------------------------------
\renewcommand\familydefault{\sfdefault}

% --- Style -------------------------------------------------------------------
\renewcommand{\bibfont}{\small \sffamily}
\renewcommand{\captionfont}{\small\sffamily}
\renewcommand{\captionlabelfont}{\bfseries}

% --- Section/SubSection/SubSubSection ----------------------------------------
\titleformat{\section}
  {\normalfont\sffamily\Large\bfseries}
  {}{0pt}{}
\titleformat{\subsection}
  {\normalfont\sffamily\large\bfseries}
  {}{0pt}{}
\titleformat{\subsubsection}
  {\normalfont\sffamily\bfseries}
  {}{0pt}{}
\titleformat*{\paragraph}
  {\sffamily\normalsize}


% --- Header / Footer ---------------------------------------------------------
\usepackage{fancyhdr}
\pagestyle{fancy}
\fancyhf{}
%\renewcommand{\headrulewidth}{0.50pt}
\renewcommand{\headrulewidth}{0pt}
\fancyhead[L]{\hspace{-0.75cm}\includegraphics[width=5.5cm]{/usr/local/share/openjournals/joss/logo.png}}
\fancyhead[C]{}
\fancyhead[R]{}
\renewcommand{\footrulewidth}{0.25pt}

\fancyfoot[L]{\parbox[t]{0.98\headwidth}{\footnotesize{\sffamily ¿citation\_author?, (2025). STCTM:
a forward modeling and retrieval framework for stellar contamination and
stellar
spectra. \textit{Journal of Open Source Software}, ¿VOL?(¿ISSUE?), ¿PAGE?. \url{https://doi.org/DOI unavailable}}}}


\fancyfoot[R]{\sffamily \thepage}
\makeatletter
\let\ps@plain\ps@fancy
\fancyheadoffset[L]{4.5cm}
\fancyfootoffset[L]{4.5cm}

% --- Macros ---------

\definecolor{linky}{rgb}{0.0, 0.5, 1.0}

\newtcolorbox{repobox}
   {colback=red, colframe=red!75!black,
     boxrule=0.5pt, arc=2pt, left=6pt, right=6pt, top=3pt, bottom=3pt}

\newcommand{\ExternalLink}{%
   \tikz[x=1.2ex, y=1.2ex, baseline=-0.05ex]{%
       \begin{scope}[x=1ex, y=1ex]
           \clip (-0.1,-0.1)
               --++ (-0, 1.2)
               --++ (0.6, 0)
               --++ (0, -0.6)
               --++ (0.6, 0)
               --++ (0, -1);
           \path[draw,
               line width = 0.5,
               rounded corners=0.5]
               (0,0) rectangle (1,1);
       \end{scope}
       \path[draw, line width = 0.5] (0.5, 0.5)
           -- (1, 1);
       \path[draw, line width = 0.5] (0.6, 1)
           -- (1, 1) -- (1, 0.6);
       }
   }

% --- Title / Authors ---------------------------------------------------------
% patch \maketitle so that it doesn't center
\patchcmd{\@maketitle}{center}{flushleft}{}{}
\patchcmd{\@maketitle}{center}{flushleft}{}{}
% patch \maketitle so that the font size for the title is normal
\patchcmd{\@maketitle}{\LARGE}{\LARGE\sffamily}{}{}
% patch the patch by authblk so that the author block is flush left
\def\maketitle{{%
  \renewenvironment{tabular}[2][]
    {\begin{flushleft}}
    {\end{flushleft}}
  \AB@maketitle}}
\makeatletter
\renewcommand\AB@affilsepx{ \protect\Affilfont}
%\renewcommand\AB@affilnote[1]{{\bfseries #1}\hspace{2pt}}
\renewcommand\AB@affilnote[1]{{\bfseries #1}\hspace{3pt}}
\renewcommand{\affil}[2][]%
   {\newaffiltrue\let\AB@blk@and\AB@pand
      \if\relax#1\relax\def\AB@note{\AB@thenote}\else\def\AB@note{#1}%
        \setcounter{Maxaffil}{0}\fi
        \begingroup
        \let\href=\href@Orig
        \let\texttt=\textttOrig
        \let\protect\@unexpandable@protect
        \def\thanks{\protect\thanks}\def\footnote{\protect\footnote}%
        \@temptokena=\expandafter{\AB@authors}%
        {\def\\{\protect\\\protect\Affilfont}\xdef\AB@temp{#2}}%
         \xdef\AB@authors{\the\@temptokena\AB@las\AB@au@str
         \protect\\[\affilsep]\protect\Affilfont\AB@temp}%
         \gdef\AB@las{}\gdef\AB@au@str{}%
        {\def\\{, \ignorespaces}\xdef\AB@temp{#2}}%
        \@temptokena=\expandafter{\AB@affillist}%
        \xdef\AB@affillist{\the\@temptokena \AB@affilsep
          \AB@affilnote{\AB@note}\protect\Affilfont\AB@temp}%
      \endgroup
       \let\AB@affilsep\AB@affilsepx
}
\makeatother
\renewcommand\Authfont{\sffamily\bfseries}
\renewcommand\Affilfont{\sffamily\small\mdseries}
\setlength{\affilsep}{1em}


\ifnum 0\ifxetex 1\fi\ifluatex 1\fi=0 % if pdftex
  \usepackage[T1]{fontenc}
  \usepackage[utf8]{inputenc}

\else % if luatex or xelatex
  \ifxetex
    \usepackage{mathspec}
    \usepackage{fontspec}

  \else
    \usepackage{fontspec}
  \fi
  \defaultfontfeatures{Ligatures=TeX,Scale=MatchLowercase}

\fi
% use upquote if available, for straight quotes in verbatim environments
\IfFileExists{upquote.sty}{\usepackage{upquote}}{}
% use microtype if available
\IfFileExists{microtype.sty}{%
\usepackage{microtype}
\UseMicrotypeSet[protrusion]{basicmath} % disable protrusion for tt fonts
}{}

\usepackage{hyperref}
\hypersetup{unicode=true,
            pdftitle={STCTM: a forward modeling and retrieval framework for stellar contamination and stellar spectra},
            pdfborder={0 0 0},
            breaklinks=true}
\urlstyle{same}  % don't use monospace font for urls

% --- We redefined \texttt, but in sections and captions we want the
% --- old definition
\let\addcontentslineOrig=\addcontentsline
\def\addcontentsline#1#2#3{\bgroup
  \let\texttt=\textttOrig\addcontentslineOrig{#1}{#2}{#3}\egroup}
\let\markbothOrig\markboth
\def\markboth#1#2{\bgroup
  \let\texttt=\textttOrig\markbothOrig{#1}{#2}\egroup}
\let\markrightOrig\markright
\def\markright#1{\bgroup
  \let\texttt=\textttOrig\markrightOrig{#1}\egroup}


\IfFileExists{parskip.sty}{%
\usepackage{parskip}
}{% else
\setlength{\parindent}{0pt}
\setlength{\parskip}{6pt plus 2pt minus 1pt}
}
\setlength{\emergencystretch}{3em}  % prevent overfull lines
\providecommand{\tightlist}{%
  \setlength{\itemsep}{0pt}\setlength{\parskip}{0pt}}
\setcounter{secnumdepth}{0}
% Redefines (sub)paragraphs to behave more like sections
\ifx\paragraph\undefined\else
\let\oldparagraph\paragraph
\renewcommand{\paragraph}[1]{\oldparagraph{#1}\mbox{}}
\fi
\ifx\subparagraph\undefined\else
\let\oldsubparagraph\subparagraph
\renewcommand{\subparagraph}[1]{\oldsubparagraph{#1}\mbox{}}
\fi

\title{STCTM: a forward modeling and retrieval framework for stellar
contamination and stellar spectra}

        \author[1, 2]{Caroline Piaulet-Ghorayeb}
    
      \affil[1]{Department of Astronomy \& Astrophysics, University of
Chicago, 5640 South Ellis Avenue, Chicago, IL 60637, USA}
      \affil[2]{E. Margaret Burbidge Prize Postdoctoral Fellow}
  \date{\vspace{-7ex}}

\begin{document}
\maketitle

\marginpar{

  \begin{flushleft}
  %\hrule
  \sffamily\small

  {\bfseries DOI:} \href{https://doi.org/DOI unavailable}{\color{linky}{DOI unavailable}}

  \vspace{2mm}

  {\bfseries Software}
  \begin{itemize}
    \setlength\itemsep{0em}
    \item \href{N/A}{\color{linky}{Review}} \ExternalLink
    \item \href{NO_REPOSITORY}{\color{linky}{Repository}} \ExternalLink
    \item \href{DOI unavailable}{\color{linky}{Archive}} \ExternalLink
  \end{itemize}

  \vspace{2mm}

  \par\noindent\hrulefill\par

  \vspace{2mm}

  {\bfseries Editor:} \href{https://example.com}{Pending
Editor} \ExternalLink \\
  \vspace{1mm}
    {\bfseries Reviewers:}
  \begin{itemize}
  \setlength\itemsep{0em}
    \item \href{https://github.com/Pending Reviewers}{@Pending
Reviewers}
    \end{itemize}
    \vspace{2mm}

  {\bfseries Submitted:} N/A\\
  {\bfseries Published:} N/A

  \vspace{2mm}
  {\bfseries License}\\
  Authors of papers retain copyright and release the work under a Creative Commons Attribution 4.0 International License (\href{http://creativecommons.org/licenses/by/4.0/}{\color{linky}{CC BY 4.0}}).

  
  \end{flushleft}
}

\hypertarget{summary}{%
\section{Summary}\label{summary}}

Transmission spectroscopy is a key avenue for the near-term study of
small-planet atmospheres and the most promising method when it comes to
searching for atmospheres on temperate rocky worlds, which are often too
cold for planetary emission to be detectable. At the same time, the
small planets that are most amenable for such atmospheric probes orbit
small M dwarf stars. This ``M-dwarf opportunity'' has encountered a
major challenge because of late-type stars' magnetic activity, which
lead to the formation of spots and faculae at their surface. If
inhomogeneously distributed throughout the photosphere, this phenomenon
can give rise to ``stellar contamination,'' or the transit light source
effect (TLSE). Specifically, the TLSE describes the fact that spectral
contrasts between bright and dark spots at the stellar surface outside
of the transit chord can leave wavelength-dependent imprints in
transmission spectra that may be mistaken for planetary atmosphere
absorption.

As the field becomes increasingly ambitious in the search for signs of
even thin atmospheres on small exoplanets, the TLSE is becoming a
limiting factor, and it becomes imperative to develop robust inference
methods to disentangle planetary and stellar contributions to the
observed spectra. Here, I present \texttt{stctm}, the STellar
ConTamination Modeling framework, a flexible Bayesian retrieval
framework to model the impact of the TLSE on any exoplanet transmission
spectrum, and infer the range of stellar surface parameters that are
compatible with the observations in the absence of any planetary
contribution. With the \texttt{exotune} sub-module, users can also
perform retrievals directly on out-of-transit stellar spectra in order
to place data-driven priors on the extent to which the TLSE can impact
any planet's transmission spectrum. The input data formats, stellar
models, and fitted parameters are easily tunable using human-readable
files and the code is fully parallelized to enable fast inferences.

\hypertarget{statement-of-need}{%
\section{Statement of need}\label{statement-of-need}}

The interpretation of high-precision exoplanet transmission spectra from
facilities such as the Hubble Space Telescope (HST) and the James Webb
Space Telescope (JWST) is increasingly dependent on a robust accounting
for the effects of stellar contamination, particularly for small planets
orbiting small stars. Despite a growing awareness of the Transit Light
Source Effect (TLSE; (Rackham et al., 2018; TRAPPIST-1 JWST Community
Initiative et al., 2024)), the community currently lacks flexible,
open-source tools that allow for robust modeling and retrieval of
stellar contamination signatures. Further, uncertainties in stellar
models motivate flexible implementations with reproducible model setups
supporting any user-specified stellar model source, such as PHOENIX or
SPHINX model grids (Husser et al., 2013; Iyer et al., 2023).

While some forward models have been developed to simulate the impact of
stellar heterogeneity on transmission spectra, these tools are either
not publicly available, computationally intractable due to their
serial-mode-only implementation, part of much larger codes that require
more advanced user training (e.g.~atmospheric retrievals), or not
designed for inference. Further, the community lacks frameworks that
enable to retrieve stellar surface properties from both observed
planetary transmission spectra and out-of-transit stellar spectra in a
Bayesian context. This gap limits our ability to quantify uncertainties
in exoplanet atmospheric properties and to test the robustness of
atmospheric detections.

\texttt{stctm} addresses this need by providing an open-source, modular,
and user-friendly framework. It allows users to model a wide range of
stellar surface configurations leveraging any spectral models, and to
infer which stellar parameters could explain observations without
invoking planetary absorption. It also supports retrievals on
out-of-transit stellar spectra to independently assess the extent of
potential stellar contamination by the host star. By enabling flexible,
fast, and reproducible inference of the TLSE, \texttt{stctm} empowers
the community to critically assess the reliability of exoplanet
atmosphere detections.

\hypertarget{main-features-of-the-code}{%
\section{Main features of the code}\label{main-features-of-the-code}}

The user inputs are communicated to the code via an input \texttt{.toml}
file for both TLSE retrievals and inferences from out-of-transit stellar
spectra. The code follows similar phases for both types of retrievals:

\begin{itemize}
\tightlist
\item
  Reading in and parsing of the inputs (data file, stellar models file,
  saving options, MCMC fit setup)
\item
  Running the MCMC fit
\item
  Post-processing to create diagnostic plots, record model comparison
  and goodness-of-fit metrics, produce publication-ready figures, and
  store sample spectra and parameters for post-processing and
  publication support and reproducibility (e.g.~Zenodo)
\end{itemize}

\texttt{exotune} retrievals on out-of-transit stellar spectra have an
additional (optional) pre-processing step, allowing users to:

\begin{itemize}
\tightlist
\item
  start from a full time-series of spectra as the input (e.g.~the output
  from Stage 3 of the \texttt{Eureka!}(Bell et al., 2022) pipeline which
  is widely used in the community), rather than simply a pre-computed
  stellar spectrum
\item
  exclude certain time intervals/exposures (e.g.~containing the transit)
  and wavelength ranges (e.g.~saturated regions) when computing the
  median spectrum to perform the retrieval over
\end{itemize}

For \texttt{exotune} retrievals, an error inflation parameter can be
leveraged to account for the often-large mismatch between the data and
stellar models.

\hypertarget{documentation}{%
\section{Documentation}\label{documentation}}

The full documentation for \texttt{stctm} with installation, testing
instructions, and real-data example retrievals on transmission spectra
and out-of-transit stellar spectra are available at
\url{https://stctm.readthedocs.io/}. A description of \texttt{stctm} can
be found in several of the early papers that employed it (see next
section).

\hypertarget{uses-of-stctm-in-the-literature}{%
\section{Uses of STCTM in the
literature}\label{uses-of-stctm-in-the-literature}}

\texttt{stctm} has been applied widely to the interpretation of
transmission spectra of rocky planets and small sub-Neptunes, including
in (Ahrer et al., 2025; Lim et al., 2023; Piaulet-Ghorayeb et al., 2025,
2024; Radica et al., 2025; Roy et al., 2023).

\hypertarget{future-developments}{%
\section{Future Developments}\label{future-developments}}

The latest version of \texttt{stctm} at the time of writing (v2.1.1)
supports MCMC retrievals on transmission spectra and on out-of-transit
stellar spectra (\texttt{exotune}), and provides model comparison
statistics, model and parameter samples as well as publication-ready
figures. Future versions will expand on these functionalities to include
user-friendly scripts tailored to post-processing only for an
already-run retrieval (creating custom plots), as well as a Nested
Sampling alternative for the retrievals. Users are encouraged to propose
or contribute any other features.

\hypertarget{similar-tools}{%
\section{Similar Tools}\label{similar-tools}}

Here are a few open-source codes that offer functionalities focused on
retrievals of the TLSE or on out-of-transit stellar spectra:

\begin{itemize}
\tightlist
\item
  Generic atmospheric retrievals (including TLSE-only retrievals on
  transmission spectra): \texttt{POSEIDON} (MacDonald \& Madhusudhan,
  2024)
\item
  Retrievals on out-of-transit stellar spectra (Nested Sampling, serial
  run mode only): \texttt{StellarFit} (Radica et al., 2025)
\end{itemize}

\hypertarget{acknowledgements}{%
\section{Acknowledgements}\label{acknowledgements}}

CPG thanks Warrick Ball for pre-review suggestions that improved the
code documentation and ease of set-up for the user.\texttt{stctm} relies
heavily on other Python libraries which include \texttt{numpy} (Harris
et al., 2020), \texttt{scipy} (Virtanen et al., 2020), \texttt{astropy}
(Astropy Collaboration et al., 2018, 2013), \texttt{matplotlib} (Hunter,
2007), \texttt{pandas}(team, 2020), \texttt{emcee}(Foreman-Mackey et
al., 2013), \texttt{corner} (Foreman-Mackey, 2016), and
\texttt{pysynphot} (Horne, 2013). Users are also strongly encouraged to
use \texttt{msg}(Townsend \& Lopez, 2023) to obtain the grids of stellar
models used in the inference step.

CPG also acknowledges support from the E. Margaret Burbidge Prize
Postdoctoral Fellowship from the Brinson Foundation. She thanks R.
MacDonald, O. Lim, and M. Radica for helpful conversations that helped
shape \texttt{stctm}.

\hypertarget{references}{%
\section*{References}\label{references}}
\addcontentsline{toc}{section}{References}

\hypertarget{refs}{}
\begin{CSLReferences}{1}{0}
\leavevmode\hypertarget{ref-Ahrer:2025}{}%
Ahrer, E.-M., Radica, M., Piaulet-Ghorayeb, C., Raul, E., Wiser, L. S.,
Welbanks, L., Acuna, L., Allart, R., Coulombe, L.-P., Louca, A. J.,
MacDonald, R. J., Saidel, M., Evans-Soma, T. M., Benneke, B., Christie,
D., Beatty, T. G., Cadieux, C., Cloutier, R., Doyon, R., \ldots{}
Schlichting, H. E. (2025). Escaping helium and a highly muted spectrum
suggest a metal-enriched atmosphere on sub-neptune GJ3090b from JWST
transit spectroscopy. \emph{arXiv e-Prints}, arXiv:2504.20428.
\url{https://doi.org/10.48550/arXiv.2504.20428}

\leavevmode\hypertarget{ref-astropy:2018}{}%
Astropy Collaboration, Price-Whelan, A. M., Sipőcz, B. M., Günther, H.
M., Lim, P. L., Crawford, S. M., Conseil, S., Shupe, D. L., Craig, M.
W., Dencheva, N., Ginsburg, A., VanderPlas, J. T., Bradley, L. D.,
Pérez-Suárez, D., de Val-Borro, M., Aldcroft, T. L., Cruz, K. L.,
Robitaille, T. P., Tollerud, E. J., \ldots{} Astropy Contributors.
(2018). {The Astropy Project: Building an Open-science Project and
Status of the v2.0 Core Package}. \emph{The Astronomical Journal},
\emph{156}(3), 123. \url{https://doi.org/10.3847/1538-3881/aabc4f}

\leavevmode\hypertarget{ref-astropy:2013}{}%
Astropy Collaboration, Robitaille, T. P., Tollerud, E. J., Greenfield,
P., Droettboom, M., Bray, E., Aldcroft, T., Davis, M., Ginsburg, A.,
Price-Whelan, A. M., Kerzendorf, W. E., Conley, A., Crighton, N.,
Barbary, K., Muna, D., Ferguson, H., Grollier, F., Parikh, M. M., Nair,
P. H., \ldots{} Streicher, O. (2013). {Astropy: A community Python
package for astronomy}. \emph{Astronomy \& Astrophysics}, \emph{558},
A33. \url{https://doi.org/10.1051/0004-6361/201322068}

\leavevmode\hypertarget{ref-bell_eureka_2022}{}%
Bell, T., Ahrer, E.-M., Brande, J., Carter, A., Feinstein, A. D.,
Caloca, G., Mansfield, M., Zieba, S., Piaulet, C., Benneke, B.,
Filippazzo, J., May, E., Roy, P.-A., Kreidberg, L., \& Stevenson, K.
(2022). {Eureka!: An End-to-End Pipeline for JWST Time-Series
Observations}. \emph{Journal of Open Source Software}, \emph{7}(79),
4503. \url{https://doi.org/10.21105/joss.04503}

\leavevmode\hypertarget{ref-corner}{}%
Foreman-Mackey, D. (2016). Corner.py: Scatterplot matrices in python.
\emph{The Journal of Open Source Software}, \emph{1}(2), 24.
\url{https://doi.org/10.21105/joss.00024}

\leavevmode\hypertarget{ref-ForemanMackey:2013}{}%
Foreman-Mackey, D., Hogg, D. W., Lang, D., \& Goodman, J. (2013). Emcee:
The MCMC hammer. \emph{Publications of the Astronomical Society of the
Pacific}, \emph{125}(925), 306. \url{https://doi.org/10.1086/670067}

\leavevmode\hypertarget{ref-harris2020array}{}%
Harris, C. R., Millman, K. J., Walt, S. J. van der, Gommers, R.,
Virtanen, P., Cournapeau, D., Wieser, E., Taylor, J., Berg, S., Smith,
N. J., Kern, R., Picus, M., Hoyer, S., Kerkwijk, M. H. van, Brett, M.,
Haldane, A., Río, J. F. del, Wiebe, M., Peterson, P., \ldots{} Oliphant,
T. E. (2020). Array programming with {NumPy}. \emph{Nature},
\emph{585}(7825), 357--362.
\url{https://doi.org/10.1038/s41586-020-2649-2}

\leavevmode\hypertarget{ref-pysynphot}{}%
Horne, K. (2013). \emph{Pysynphot: Synthetic photometry software}.
\url{https://ascl.net/1303.023}; Astrophysics Source Code Library.

\leavevmode\hypertarget{ref-Hunter:2007}{}%
Hunter, J. D. (2007). Matplotlib: A 2D graphics environment.
\emph{Computing in Science \& Engineering}, \emph{9}(3), 90--95.
\url{https://doi.org/10.1109/MCSE.2007.55}

\leavevmode\hypertarget{ref-Husser:2013}{}%
Husser, T.-O., Wende-von Berg, S., Dreizler, S., Homeier, D., Reiners,
A., Barman, T., \& Hauschildt, P. H. (2013). A new extensive library of
PHOENIX stellar atmospheres and synthetic spectra. \emph{Astronomy \&
Astrophysics}, \emph{553}, A6.
\url{https://doi.org/10.1051/0004-6361/201219058}

\leavevmode\hypertarget{ref-Iyer:2023}{}%
Iyer, A. R., Line, M. R., Muirhead, P. S., Fortney, J. J., \&
Gharib-Nezhad, E. (2023). The SPHINX m-dwarf spectral grid. I.
Benchmarking new model atmospheres to derive fundamental m-dwarf
properties. \emph{The Astrophysical Journal}, \emph{944}(1), 41.
\url{https://doi.org/10.3847/1538-4357/acabc2}

\leavevmode\hypertarget{ref-Lim2023}{}%
Lim, O., Benneke, B., Doyon, R., MacDonald, R. J., Piaulet, C., Artigau,
É., Coulombe, L.-P., Radica, M., L'Heureux, A., Albert, L., Rackham, B.
V., de Wit, J., Salhi, S., Roy, P.-A., Flagg, L., Fournier-Tondreau, M.,
Taylor, J., Cook, N. J., Lafrenière, D., \ldots{} Darveau-Bernier, A.
(2023). {Atmospheric Reconnaissance of TRAPPIST-1 b with JWST/NIRISS:
Evidence for Strong Stellar Contamination in the Transmission Spectra}.
\emph{The Astrophysical Journal Letters}, \emph{955}(1), L22.
\url{https://doi.org/10.3847/2041-8213/acf7c4}

\leavevmode\hypertarget{ref-poseidon}{}%
MacDonald, R. J., \& Madhusudhan, N. (2024). \emph{POSEIDON:
Multidimensional atmospheric retrieval of exoplanet spectra}.
\url{https://doi.org/10.21105/joss.04873}

\leavevmode\hypertarget{ref-PiauletGhorayeb:2024}{}%
Piaulet-Ghorayeb, C., Benneke, B., Radica, M., Raul, E., Coulombe,
L.-P., Ahrer, E.-M., Kubyshkina, D., Howard, W. S., Krissansen-Totton,
J., MacDonald, R. J., Roy, P.-A., Louca, A., Christie, D.,
Fournier-Tondreau, M., Allart, R., Miguel, Y., Schlichting, H. E.,
Welbanks, L., Cadieux, C., \ldots{} Knutson, H. A. (2024). JWST/NIRISS
reveals the water-rich {``steam world''} atmosphere of GJ 9827 d.
\emph{The Astrophysical Journal Letters}, \emph{974}(1), L10.
\url{https://doi.org/10.3847/2041-8213/ad6f00}

\leavevmode\hypertarget{ref-PiauletGhorayeb:2025}{}%
Piaulet-Ghorayeb, C., Benneke, B., Turbet, M., Moore, K., Roy, P.-A.,
Lim, O., Doyon, R., Fauchez, T. J., Albert, L., Radica, M., Coulombe,
L.-P., Lafrenière, D., Cowan, N. B., Belzile, D., Musfirat, K., Kaur,
M., L'Heureux, A., Johnstone, D., MacDonald, R. J., \ldots{} Turner, J.
D. (2025). Strict limits on potential secondary atmospheres on the
temperate rocky exo-earth TRAPPIST-1 d. \emph{arXiv e-Prints}.
\url{https://ui.adsabs.harvard.edu/abs/2024ApJ...974L..10P}

\leavevmode\hypertarget{ref-Rackham:2018}{}%
Rackham, B. V., Apai, D., \& Giampapa, M. S. (2018). The transit light
source effect: False spectral features and incorrect densities for
m-dwarf transiting planets. \emph{The Astrophysical Journal},
\emph{853}(2), 122. \url{https://doi.org/10.3847/1538-4357/aaa08c}

\leavevmode\hypertarget{ref-Radica:2025}{}%
Radica, M., Piaulet-Ghorayeb, C., Taylor, J., Coulombe, L.-P., Benneke,
B., Albert, L., Artigau, É., Cowan, N. B., Doyon, R., Lafrenière, D.,
L'Heureux, A., \& Lim, O. (2025). Promise and peril: Stellar
contamination and strict limits on the atmosphere composition of
TRAPPIST-1 c from JWST NIRISS transmission spectra. \emph{The
Astrophysical Journal Letters}, \emph{979}(1), L5.
\url{https://doi.org/10.3847/2041-8213/ada381}

\leavevmode\hypertarget{ref-Roy:2023}{}%
Roy, P.-A., Benneke, B., Piaulet, C., Gully-Santiago, M. A., Crossfield,
I. J. M., Morley, C. V., Kreidberg, L., Mikal-Evans, T., Brande, J.,
Delisle, S., Greene, T. P., Hardegree-Ullman, K. K., Barman, T.,
Christiansen, J. L., Dragomir, D., Fortney, J. J., Howard, A. W.,
Kosiarek, M. R., \& Lothringer, J. D. (2023). Water absorption in the
transmission spectrum of the water world candidate GJ 9827 d. \emph{The
Astrophysical Journal Letters}, \emph{954}(2), L52.
\url{https://doi.org/10.3847/2041-8213/acebf0}

\leavevmode\hypertarget{ref-pandas:2020}{}%
team, T. pandas development. (2020). Pandas-dev/pandas: pandas.
\emph{Zenodo}. \url{https://doi.org/10.5281/zenodo.3509134}

\leavevmode\hypertarget{ref-Townsend:2023}{}%
Townsend, R., \& Lopez, A. (2023). MSG: A software package for
interpolating stellar spectra in pre-calculated grids. \emph{The Journal
of Open Source Software}, \emph{8}(81), 4602.
\url{https://doi.org/10.21105/joss.04602}

\leavevmode\hypertarget{ref-TRAPPIST1_JWST_2024}{}%
TRAPPIST-1 JWST Community Initiative, Wit, J. de, Doyon, R., Rackham, B.
V., Lim, O., Ducrot, E., Kreidberg, L., Benneke, B., Ribas, I., Berardo,
D., Niraula, P., Iyer, A., Shapiro, A., Kostogryz, N., Witzke, V.,
Gillon, M., Agol, E., Meadows, V., Burgasser, A. J., \ldots{} Way, M. J.
(2024). A roadmap for the atmospheric characterization of terrestrial
exoplanets with JWST. \emph{Nature Astronomy}, \emph{8}, 810--818.
\url{https://doi.org/10.1038/s41550-024-02298-5}

\leavevmode\hypertarget{ref-2020SciPy-NMeth}{}%
Virtanen, P., Gommers, R., Oliphant, T. E., Haberland, M., Reddy, T.,
Cournapeau, D., Burovski, E., Peterson, P., Weckesser, W., Bright, J.,
van der Walt, S. J., Brett, M., Wilson, J., Millman, K. J., Mayorov, N.,
Nelson, A. R. J., Jones, E., Kern, R., Larson, E., \ldots{} SciPy 1.0
Contributors. (2020). {{SciPy} 1.0: Fundamental Algorithms for
Scientific Computing in Python}. \emph{Nature Methods}, \emph{17},
261--272. \url{https://doi.org/10.1038/s41592-019-0686-2}

\end{CSLReferences}

\end{document}
